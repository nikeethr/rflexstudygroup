%        File: newfile.tex
%     Created: Sun Sep 24 08:00 PM 2017 AEST
% Last Change: Sun Sep 24 08:00 PM 2017 AEST
%
\documentclass[a4paper]{article}

\usepackage[sumlimits,]{amsmath}
\usepackage{amssymb}
\usepackage{parskip}

\begin{document}
Conjecture: Prove that the only prime triplets (i.e. three primes, each 2 from the next) is 3,5,7

From question 5. we know that for any integer $n$, at least one of $n$, $n+2$, $n+4$ is divisible by
3.

By the definition of a prime number, this means that any three integers separated by $2$ cannot be
prime unless one of $n$, $n+2$, $n+4$ is $3$.

If $n = 3$,\\
$n=3$, $n+2=5$, $n+4=7$, which is a prime triple which agrees with the conjecture.\\
If $n+2 = 3$,\\
$n=1$, $n+2=3$, $n+4=5$, 1 is not prime, so this is not a prime triple.\\
If $n+4 = 3$,\\
$n=-1$, $n+2=1$, $n+4=3$, 1 and -1 are not prime, so this is not a prime triple either.

So, we have shown that the only possible prime triple is 3,5,7.

\end{document}
