%        File: newfile.tex
%     Created: Sun Sep 24 08:00 PM 2017 AEST
% Last Change: Sun Sep 24 08:00 PM 2017 AEST
%
\documentclass[a4paper]{article}

\usepackage[sumlimits,]{amsmath}
\usepackage{amssymb}
\usepackage{parskip}

\begin{document}
Conjecture: For any integer $n$, at least one of the integers $n$, $n+2$, $n+4$ is divisible by $3$.
\\
Assume the conjecture is false. Then by the division theorem,
$\exists q_1,q_2,q_3,r_1,r_2,r_3 \in \mathcal{Z}, 0 < r_1,r_2,r_3 < 3$ such that
\begin{align}
  n &= 3q_1 + r_1 \label{eqn:n1} \\
  n + 2 &= 3q_2 + r_2 \label{eqn:n2} \\
  n + 4 &= 3q_3 + r_3 \label{eqn:n3}
\end{align}

given that $r_1$ can only be $1$ or $2$, when $r_1 = 1$, (\ref{eqn:n2}) becomes
\begin{align*}
  3q_1 + 3 = 3q_2 + r_2 && \text{Substituting (\ref{eqn:n1}) and $r_1 = 1$} \\
  r_2 = 3(q_1 - q_2 + 1) && \text{Re-arranging}
\end{align*}

Therefore, $n+2$ is divisible by 3 when $r_1 = 1$. Which is a contradiction.\\\\
When $r_1 = 2$, (\ref{eqn:n3}) becomes
\begin{align*}
  3q_1 + 6 = 3q_3 + r_2 && \text{Substituting (\ref{eqn:n1}) and $r_1 = 2$} \\
  r_2 = 3(q_1 - q_3 + 2) && \text{Re-arranging}
\end{align*}
Therefore, $n+4$ is divisible by 3 when $r_1 = 2$. Which is also a contradiction.\\\\
We have shown that for all possible values of $r_1$ such that $n$ is not divisible by $3$ either
$n+2$ or $n+4$ is divisible by three. The only other case is that $r_1$ is zero and $n$ is divisible
by $3$.
Therefore, our initial assumption is false and the conjecture must be true.
\end{document}
