%        File: newfile.tex
%     Created: Sun Sep 24 08:00 PM 2017 AEST
% Last Change: Sun Sep 24 08:00 PM 2017 AEST
%
\documentclass[a4paper]{article}

\usepackage[sumlimits,]{amsmath}
\usepackage{amssymb}
\usepackage{parskip}

\begin{document}
Give an example of family of intervals $A_n, n = 1,2,\dots$, such that $A_{n+1} \subset A_{n}$ for
all n and $\cap_{n=1}^{\infty}A_n = 1$.

Let $A_n$ be the interval $[\frac{n}{n+1}, 1]$,
then $A_{n+1}$ is the interval $[\frac{n+1}{n+2}, 1]$

The greatest lower bound of $A_n$ is $\frac{n}{n+1}$ and the greatest lower bound of $A_{n+1}$ is
$\frac{n+1}{n+2}$. They have the same least upper bound, $1$.
\begin{align*}
  \frac{n}{n+1} &= \frac{n}{n+1}\frac{n+2}{n+2} && \text{Multiply and divide by n+2} \\
  &= \frac{n^2 + n}{(n+1)(n+2)} && \text{Expand out numerator} \\
  &< \frac{n^2 + n + 1}{(n+1)(n+2)} && \text{Add 1 to numerator} \\
  &= \frac{(n+1)^2}{(n+1)(n+2)} && \text{Factor out $n+1$ from numerator} \\
  &= \frac{n+1}{n+2} && \text{Cancel out $n+1$}
\end{align*}

Therefore $A_{n+1} \subset A_n$ since the greatest lower bound of $A_n$ is strictly less than the
greatest lower bound of $A_{n+1}$.

Now as $n \to \infty$, $\frac{n}{n+1} \to 1$, so $A_n \to [1, 1] = \left\{\,1\right\}$ since the
greatest lower bound and least upper bound of $[1, 1]$ are both $1$. As $1$ is contained in all
$A_n$, 
$\cap_{n=1}^{\infty}A_n = 1$. The family of intervals that lead to this conclusion was the
interval $[\frac{n}{n+1}, 1],\, n=1,2,\dots$
\end{document}
