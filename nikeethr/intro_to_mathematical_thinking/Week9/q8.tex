%        File: newfile.tex
%     Created: Sun Sep 24 08:00 PM 2017 AEST
% Last Change: Sun Sep 24 08:00 PM 2017 AEST
%
\documentclass[a4paper]{article}

\usepackage[sumlimits,]{amsmath}
\usepackage{amssymb}
\usepackage{parskip}

\begin{document}
Conjecture: If the sequence $\left\{ a_n \right\}_{n=1}^{\infty}$ tends to limit $L$ as $n \to
\infty$. then for any fixed number $M > 0$, the sequence $\left\{ Ma_n \right\}_{n=1}^{\infty}$
tends to the limit $ML$.

Since $\{a_n\}_{n=1}^{\infty} \to L$ as $n \to \infty$, by the definition of the limit for all $\delta
> 0$ there exists $N \in \mathcal{N}$ such that $|a_n - L| < \delta$ for $n > N$.

For any $\epsilon > 0$, we can choose $\epsilon = M\delta$. So, $|a_n - L| < \epsilon / M$ for some
fixed $M$.
Then for some $n > N$,
\begin{equation}
  |Ma_n - ML| = M|a_n - L| < M \frac{\epsilon}{M} = \epsilon
\end{equation}

By definition this means that $\{Ma_n\}_{n=1}^{\infty} \to ML$ as $n \to \infty$ and the conjecture
is true.

\end{document}
