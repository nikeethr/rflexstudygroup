%        File: newfile.tex
%     Created: Sun Sep 24 08:00 PM 2017 AEST
% Last Change: Sun Sep 24 08:00 PM 2017 AEST
%
\documentclass[a4paper]{article}
\usepackage[sumlimits,]{amsmath}
\usepackage{amssymb}

\begin{document}
\begin{equation}
  (\exists m \in \mathcal{N})(\exists n \in \mathcal{N})(3m + 5n = 12)
  \label{eqn:proof}
\end{equation}
We will prove that (\ref{eqn:proof}) is false by showing that $\forall n \in \mathcal{N}$ there
$\nexists m \in \mathcal{N}$ such that $3m + 5n = 12$. \\\\
Since $n \in \mathcal{N}$, $n\geq1$ therefore $5n\geq5$. This means that $5 \leq 5n = 12 - 3m$. If $m >
2$ then $12 - 3m < 5$ so $m$ must be either $1$ or $2$. \\\\
If $m = 1$, \\
$5n = 9$. 9 is not divisible by 5 therefore $n \not\in \mathcal{N}$ which is a contradiction so $m
\neq 1$ \\\\
If $m = 2$, \\
$5n = 6$. 6 is not divisible by 5 therefore $n \not\in \mathcal{N}$ which is also a contradiction so
$m \neq
2$ \\\\
Thus, $\nexists m \in \mathcal{N}$ such that (\ref{eqn:proof}) is true. so (\ref{eqn:proof}) must be
false.
\end{document}
